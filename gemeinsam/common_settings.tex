%----------------------------------------------------------------------------------------
%	PAKETE UND THEMEN
%----------------------------------------------------------------------------------------

\documentclass{beamer}

\mode<presentation> {

% Die Beamer-Klasse wird mit einer Reihe von Standard Folienthemen,
% mit unterschiedlichen Farben und Folienlayouts. Nachfolgend eine Liste aller verfügbaren Themen.
% Hier kann das jeweils gewünschte Thema unkommentiert werden.

%\usetheme{default}
% \usetheme{AnnArbor}
% \usetheme{Antibes}
% \usetheme{Bergen}
% \usetheme{Berkeley}
% \usetheme{Berlin}
% \usetheme{Boadilla}
% \usetheme{CambridgeUS}
% \usetheme{Copenhagen}
% \usetheme{Darmstadt}
% \usetheme{Dresden}
% \usetheme{Frankfurt}
% \usetheme{Goettingen}
% \usetheme{Hannover}
% \usetheme{Ilmenau}
% \usetheme{JuanLesPins}
% \usetheme{Luebeck}
 \usetheme{Madrid}
% \usetheme{Malmoe}
% \usetheme{Marburg}
% \usetheme{Montpellier}
% \usetheme{PaloAlto}
% \usetheme{Pittsburgh}
% \usetheme{Rochester}
% \usetheme{Singapore}
% \usetheme{Szeged}
% \usetheme{Warsaw}

% Neben verschiedenen Themen kommt die Beamer-Klasse auch mit diversen Farbvarianten
% für jedes Folienthema daher. Hier kann das jeweils gewünschte Farbthema unkommentiert werden.

% \usecolortheme{albatross}
% \usecolortheme{beaver}
% \usecolortheme{beetle}
% \usecolortheme{crane}
\usecolortheme{dolphin}
% \usecolortheme{dove}
% \usecolortheme{fly}
% \usecolortheme{lily}
% \usecolortheme{orchid}
% \usecolortheme{rose}
% \usecolortheme{seagull}
% \usecolortheme{seahorse}
% \usecolortheme{whale}
% \usecolortheme{wolverine}

%----------------------------------------------------------------------------------------
%	DEFINITION BENUTZERDEFINIERTER FARBEN (RGB)
%----------------------------------------------------------------------------------------
\definecolor{dunkelgrau}{rgb}{0.8,0.8,0.8}
\definecolor{hellgrau}{rgb}{0.95,0.95,0.95}
\definecolor{hellblau}{rgb}{0.8,0.85,1}
\definecolor{dunkelblau}{rgb}{0,0,0.9}

% \setbeamertemplate{footline}				% Zum Entfernen der Fusszeile auf allen Folien kann diese Zeile unkommentiert werden
\setbeamertemplate{footline}[frame number]		% Zum Ersetzen der Fußzeile mit einem einfachen Folienzähler kann diese Zeile auskommentiert werden

\setbeamertemplate{navigation symbols}{}		% Zum Anzeigen der Navigationssymbole am unteren Rand aller Folien kann diese Zeile auskommentiert werden
}

\setbeamertemplate{background canvas} [vertical shading][top=hellblau, bottom=white] 

% \usepackage{pgfpages}				% Zum Erstellen vernünftiger Ausdrucke/Handouts diese Zeile(n) unkommentieren
% \pgfpagesuselayout{resize to}[a4paper,border shrink=5mm,landscape]	% Einseitiger Ausdruck auf A4
% \pgfpagesuselayout{2 on 1}[a4paper,border shrink=5mm]			% Druckt 2 Folien auf 1 Seite A4
% \pgfpagesuselayout{4 on 1}[a4paper,border shrink=5mm,landscape]		% Druckt 4 Folien auf 1 Seite A4
% \pgfpagesuselayout{8 on 1}[a4paper,border shrink=5mm]			% Druckt 8 Folien auf 1 Seite A4
% \pgfpagesuselayout{16 on 1}[a4paper,border shrink=5mm,landscape]		% Druckt 16 Folien auf 1 Seite A4

% \setbeameroption{show notes on second screen=right}	% Zum Anzeigen von Notizen auf dem 2. Schirm.  Mit <location> = left, right, bottom oder top
									% Benötigt das Paket pgfpages!

%\beamerdefaultoverlayspecification{<+->}		% Aufzählungen immer schrittweise zeigen

% \setbeamercovered{transparent}			% Sorgt dafür, daß versteckte Textteile nicht unsichtbar, sondern mit wenig Kontast dargestellt werden

\setbeamerfont{frametitle}{family=\sffamily,series=\bfseries,size={\fontsize{18.00}{20.00}}}	% Setzen der Schriftart und -größe für die Überschriften
\setbeamerfont{footline}{family=\sffamily,series=\bfseries,size={\fontsize{6.00}{7.00}}}		% Setzen der Schriftart und -größe für die Fusszeile

\usepackage[ngerman]{babel}				% für Deutsch
%\usepackage{ngerman}
%\usepackage[english]{babel}				% für Englisch
%\usepackage[latin1]{inputenc}
\usepackage[utf8]{inputenc}
\usepackage{graphicx}					% Erlaubt das Einbinden von Grafiken/Bildern
\usepackage{eurosym}					% Erlaubt das Anzeigen des Eurosymbols
\usepackage{xcolor}					% Erlaubt den Einsatz benutzerdefinierter Farben
								% Weitere Einsatzmöglichkeiten:
								% \textcolor{Farbe}{Text} entspricht Befehlen, wie \textbf{}
								% \pagecolor{Name}, \pagecolor{model}{specification} setzt die Hintergrundfarbe der Seite
								% \colorbox{Name}{Text}, \colorbox{model}{specification}{Text} hinterlegt den Text mit der Farbe
								% \fcolorbox{Name}{Text}, \fcolorbox{model}{specification}{Text} umrandet den Text mit der Farbe
								% vordefinierte Farben: black, white, red, green, blue, cyan, magenta, yellow

\usepackage{colortbl}					% Erlaubt das Verwenden benutzerdefinierter Farben in Tabellen
								% \cellcolor{farbe}, \rowcolor{farbe} oder \columncolor{farbe}

\usepackage{url}						% Erlaubt das Verwenden von URLs ohne umständliche Maskierung
\urlstyle{tt}							% Durch Verwendung von \urlstyle{tt} kann zusätzlich die Schriftart auf Typewriter gestellt werden
\urldef{\fsog}{\url}{http://www.FreieSoftwareOG.org}	% Häufig vorkommende URLs können in einem Makro hinterlegt werden.

\usepackage{booktabs}					% Erlaubt den Einsatz von \toprule, \midrule und \bottomrule innerhalb von Tabellen

% \usepackage{fontspec}					% Erlaubt das Anpassen von Schriftarten. Z.B. auch um eigene Fonts zu verwenden. MUSS MIT XELATEX KOMPILIERT WERDEN!!
% \defaultfontfeatures{Mapping=tex-text}
% \setmainfont{Yanone Kaffeesatz}			% (Haupt-)Schriftart der Präsentation

% Will man ein globales Hintergrundbild verwenden, muss dies in der Preamble definiert werden
% Hier verwende ich das der FSOG
%\usebackgroundtemplate%
%{
%  \includegraphics[width=\paperwidth,height=\paperheight]{../gemeinsam/Praesentationshintergrund}%
%}

\usepackage{pgf}  
\logo{\pgfputat{\pgfxy(-1.2,-0.2)}{\pgfbox[center,base]{\includegraphics[width=2.6cm]{../gemeinsam/fsog.png}}}}  

%\logo{\includegraphics[width=3cm]{../gemeinsam/fsog.png}}


