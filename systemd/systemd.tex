%%%%%%%%%%%%%%%%%%%%%%%%%%%%%%%%%%%%%%%%%
% Beamer Präsentation
% LaTeX Vorlage
% Version 1.0 (10/11/12)
%
% Diese Vorlage wurde heruntergeladen von:
% http://www.LaTeXTemplates.com
%
% Lizenz:
% CC BY-NC-SA 3.0 (http://creativecommons.org/licenses/by-nc-sa/3.0/)
%
%%%%%%%%%%%%%%%%%%%%%%%%%%%%%%%%%%%%%%%%%

%----------------------------------------------------------------------------------------
%	PAKETE UND THEMEN
%----------------------------------------------------------------------------------------

\documentclass{beamer}

\mode<presentation> {

% Die Beamer-Klasse wird mit einer Reihe von Standard Folienthemen,
% mit unterschiedlichen Farben und Folienlayouts. Nachfolgend eine Liste aller verfügbaren Themen.
% Hier kann das jeweils gewünschte Thema unkommentiert werden.

%\usetheme{default}
% \usetheme{AnnArbor}
% \usetheme{Antibes}
% \usetheme{Bergen}
% \usetheme{Berkeley}
% \usetheme{Berlin}
% \usetheme{Boadilla}
% \usetheme{CambridgeUS}
% \usetheme{Copenhagen}
% \usetheme{Darmstadt}
% \usetheme{Dresden}
% \usetheme{Frankfurt}
% \usetheme{Goettingen}
% \usetheme{Hannover}
% \usetheme{Ilmenau}
% \usetheme{JuanLesPins}
% \usetheme{Luebeck}
 \usetheme{Madrid}
% \usetheme{Malmoe}
% \usetheme{Marburg}
% \usetheme{Montpellier}
% \usetheme{PaloAlto}
% \usetheme{Pittsburgh}
% \usetheme{Rochester}
% \usetheme{Singapore}
% \usetheme{Szeged}
% \usetheme{Warsaw}

% Neben verschiedenen Themen kommt die Beamer-Klasse auch mit diversen Farbvarianten
% für jedes Folienthema daher. Hier kann das jeweils gewünschte Farbthema unkommentiert werden.

% \usecolortheme{albatross}
% \usecolortheme{beaver}
% \usecolortheme{beetle}
% \usecolortheme{crane}
\usecolortheme{dolphin}
% \usecolortheme{dove}
% \usecolortheme{fly}
% \usecolortheme{lily}
% \usecolortheme{orchid}
% \usecolortheme{rose}
% \usecolortheme{seagull}
% \usecolortheme{seahorse}
% \usecolortheme{whale}
% \usecolortheme{wolverine}

%----------------------------------------------------------------------------------------
%	DEFINITION BENUTZERDEFINIERTER FARBEN (RGB)
%----------------------------------------------------------------------------------------
\definecolor{dunkelgrau}{rgb}{0.8,0.8,0.8}
\definecolor{hellgrau}{rgb}{0.95,0.95,0.95}
\definecolor{hellblau}{rgb}{0.8,0.85,1}
\definecolor{dunkelblau}{rgb}{0,0,0.9}

% \setbeamertemplate{footline}				% Zum Entfernen der Fusszeile auf allen Folien kann diese Zeile unkommentiert werden
\setbeamertemplate{footline}[frame number]		% Zum Ersetzen der Fußzeile mit einem einfachen Folienzähler kann diese Zeile auskommentiert werden

\setbeamertemplate{navigation symbols}{}		% Zum Anzeigen der Navigationssymbole am unteren Rand aller Folien kann diese Zeile auskommentiert werden
}

\setbeamertemplate{background canvas} [vertical shading][top=hellblau, bottom=white] 

% \usepackage{pgfpages}				% Zum Erstellen vernünftiger Ausdrucke/Handouts diese Zeile(n) unkommentieren
% \pgfpagesuselayout{resize to}[a4paper,border shrink=5mm,landscape]	% Einseitiger Ausdruck auf A4
% \pgfpagesuselayout{2 on 1}[a4paper,border shrink=5mm]			% Druckt 2 Folien auf 1 Seite A4
% \pgfpagesuselayout{4 on 1}[a4paper,border shrink=5mm,landscape]		% Druckt 4 Folien auf 1 Seite A4
% \pgfpagesuselayout{8 on 1}[a4paper,border shrink=5mm]			% Druckt 8 Folien auf 1 Seite A4
% \pgfpagesuselayout{16 on 1}[a4paper,border shrink=5mm,landscape]		% Druckt 16 Folien auf 1 Seite A4

% \setbeameroption{show notes on second screen=right}	% Zum Anzeigen von Notizen auf dem 2. Schirm.  Mit <location> = left, right, bottom oder top
									% Benötigt das Paket pgfpages!

%\beamerdefaultoverlayspecification{<+->}		% Aufzählungen immer schrittweise zeigen

% \setbeamercovered{transparent}			% Sorgt dafür, daß versteckte Textteile nicht unsichtbar, sondern mit wenig Kontast dargestellt werden

\setbeamerfont{frametitle}{family=\sffamily,series=\bfseries,size={\fontsize{18.00}{20.00}}}	% Setzen der Schriftart und -größe für die Überschriften
\setbeamerfont{footline}{family=\sffamily,series=\bfseries,size={\fontsize{6.00}{7.00}}}		% Setzen der Schriftart und -größe für die Fusszeile

\usepackage[ngerman]{babel}				% für Deutsch
%\usepackage{ngerman}
%\usepackage[english]{babel}				% für Englisch
%\usepackage[latin1]{inputenc}
\usepackage[utf8]{inputenc}
\usepackage{graphicx}					% Erlaubt das Einbinden von Grafiken/Bildern
\usepackage{eurosym}					% Erlaubt das Anzeigen des Eurosymbols
\usepackage{xcolor}					% Erlaubt den Einsatz benutzerdefinierter Farben
								% Weitere Einsatzmöglichkeiten:
								% \textcolor{Farbe}{Text} entspricht Befehlen, wie \textbf{}
								% \pagecolor{Name}, \pagecolor{model}{specification} setzt die Hintergrundfarbe der Seite
								% \colorbox{Name}{Text}, \colorbox{model}{specification}{Text} hinterlegt den Text mit der Farbe
								% \fcolorbox{Name}{Text}, \fcolorbox{model}{specification}{Text} umrandet den Text mit der Farbe
								% vordefinierte Farben: black, white, red, green, blue, cyan, magenta, yellow

\usepackage{colortbl}					% Erlaubt das Verwenden benutzerdefinierter Farben in Tabellen
								% \cellcolor{farbe}, \rowcolor{farbe} oder \columncolor{farbe}

\usepackage{url}						% Erlaubt das Verwenden von URLs ohne umständliche Maskierung
\urlstyle{tt}							% Durch Verwendung von \urlstyle{tt} kann zusätzlich die Schriftart auf Typewriter gestellt werden
\urldef{\fsog}{\url}{http://www.FreieSoftwareOG.org}	% Häufig vorkommende URLs können in einem Makro hinterlegt werden.

\usepackage{booktabs}					% Erlaubt den Einsatz von \toprule, \midrule und \bottomrule innerhalb von Tabellen

% \usepackage{fontspec}					% Erlaubt das Anpassen von Schriftarten. Z.B. auch um eigene Fonts zu verwenden. MUSS MIT XELATEX KOMPILIERT WERDEN!!
% \defaultfontfeatures{Mapping=tex-text}
% \setmainfont{Yanone Kaffeesatz}			% (Haupt-)Schriftart der Präsentation

% Will man ein globales Hintergrundbild verwenden, muss dies in der Preamble definiert werden
% Hier verwende ich das der FSOG
%\usebackgroundtemplate%
%{
%  \includegraphics[width=\paperwidth,height=\paperheight]{../gemeinsam/Praesentationshintergrund}%
%}

\usepackage{pgf}  
\logo{\pgfputat{\pgfxy(-1.2,-0.2)}{\pgfbox[center,base]{\includegraphics[width=2.6cm]{../gemeinsam/fsog.png}}}}  

%\logo{\includegraphics[width=3cm]{../gemeinsam/fsog.png}}




%----------------------------------------------------------------------------------------
% DEFINITION DER TITELFOLIE
%----------------------------------------------------------------------------------------

\title[section77 e.V. - systemd]{systemd} % Der Kurztitel (in eckiger Klammer) erscheint im unteren Bereich aller Folien, der Haupttitel (in geschweifter Klammer) erscheint nur auf der Titelfolie

\author{von Justin und Andy}        % Name des Vortragenden
\institute[section77 e.V.]    % Die Institution. Erscheint am unteren Rand aller Folien, zum Platz sparen möglichst eine Abkürzung
{
Community section77 e.V. \\  % Die Institution. Erscheint auf der Titelfolie
\medskip
}
\date{\today}                 % Das Datum des Vortrags (hier: aktuelles Datum)

%----------------------------------------------------------------------------------------
% ANFANG DES DOKUMENTES / DER PRÄSENTATION
%----------------------------------------------------------------------------------------

\begin{document}

\begin{frame}      % Der Zusatz [plain] sorgt dafür, daß keine Seitenzahl/Fusszeile verwendet wird
  \titlepage        % Ausgabe der Titelfolie als erste Folie
\end{frame}

% \begin{frame}         % Inhaltsfolie. Zum Anzeigen können diese Zeilen (von \begin bis \end) unkommentiert werden
% \frametitle{Inhalt der Präsentation}  % Wenn innerhalb der Präsentation \section{} und \subsection{} Befehle verwendet werden,
% \tableofcontents        % werden diese automatisch in diese Inhaltsfolie aufgenommen
% \end{frame}

%------------------------------------------------------------------------------------------
% AB HIER BEGINNT DIE EIGENTLICHE PRÄSENTATION
%------------------------------------------------------------------------------------------

\begin{frame}
\frametitle{was ist systemd?}
ein Daemon, der als Init-Prozess - als erster Prozess - zum Starten, Überwachen
und Beenden weiterer Prozesse dient.

In den Standard-Distros inzwischen enthalten und dort zum Default geworden:

\begin{itemize}
  \item SLES (seit 12)
  \item Debian (seit 8)
  \item Ubuntu (seit 15.04)
  \item Fedora (seit 15)
  \item OpenSuse (seit 12.1)
  \item RHEL (seit 7)
  \item Arch (seit Okt. 2012)
\end{itemize}

\end{frame}

\begin{frame}
  \frametitle{Vorteile}
  \begin{itemize}
    \item höhere Geschwindigkeit beim Start des Systems durch parallel gestartete
          Prozesse. Dies macht besonders bei Mehrkernprozessoren Sinn
    \item systemd kümmert sich um eine Strategie wie die Prozesse nacheinander
          gestartet werden können. Der Anwender gibt einfach nur an welche anderen Prozesse benötigt werden
    \item vereinfachte Schnittstelle zur Steuerung der Dienste und Betrachtung der Logfiles
  \end{itemize}
\end{frame}

\begin{frame}
  \frametitle{Nachteile}
  \begin{itemize}
    \item bricht das Basiskonzept "`ein Programm soll ein Problem lösen, dieses aber möglichst gut"'.
    \item speichert Logdateien im Binärformat, nicht als Text
    \item systemd wurde schon mehrfach nachgesagt, es spalte die Community (Stichwort Devuan). Teilweise richtet
  sich dies direkt gegen die zwei Haupentwickler. Auch Linus Torvalds hat sich schon zu Wort gemeldet (0day patch)
  \end{itemize}

\end{frame}

\begin{frame}[fragile]

\frametitle{Vergleich: Kommando bei init/systemd}

  Start eines Dienstes
  \begin{itemize}
    \item init:
    \begin{verbatim}
    $ /etc/init.d/apache2 start
    \end{verbatim}

    \item systemd:
    \begin{verbatim}
    $ systemctl start apache2
    \end{verbatim}
  \end{itemize}

\end{frame}

\begin{frame}[fragile]
\frametitle{Vergleich: Kommando bei init/systemd}

  Aktivieren eines Dienstes (automatischer Start beim Boot):
  \begin{itemize}

    \item RedHat-Style:
    \begin{verbatim}
    $ chkconfig 235 httpd
    \end{verbatim}

    \item SysVinit-Style:
    \begin{verbatim}
    $ ln -s /etc/init.d/apache2 /etc/rc2.d
    # (und das Ganze nochmal für die runlevel 3 und 5)
    \end{verbatim}

    \item systemd:
    \begin{verbatim}
    $ systemctl enable apache2
    \end{verbatim}

  \end{itemize}

\end{frame}


\begin{frame}[fragile]
\frametitle{Status/Logging}

  \begin{itemize}

    \item Ohne systemd:
    \begin{verbatim}
    $ less /var/log/...
    \end{verbatim}
    oder mit grep in den logfiles suchen

    \item Mit systemd:
    \begin{verbatim}
    $ systemctl status apache2
    $ journalctl -u apache2
    $ journalctl -p err..alert
    \end{verbatim}

  \end{itemize}

\end{frame}

\begin{frame}[fragile]
\frametitle{Ein einfacher service}

  Ein Skript, welches alle 15s die Außentemperatur bei Justin abfrägt (get\_temp.sh)
  soll beim Systemstart automatisch im Hintergrund gestartet werden:

\begin{verbatim}
[Unit]
Description=read outdoor temperature service
After=systemd-user-sessions.service

[Service]
Type=simple
User=andy
ExecStart=/home/andy/src/fsog_latex_beamer/systemd/get_temp.sh

[Install]
WantedBy=multi-user.target
\end{verbatim}

\end{frame}

\begin{frame}[fragile]

An die richtige Stelle kopieren und starten
\begin{verbatim}
$ sudo cp get_temp.service /etc/systemd/system
$ sudo systemctl start get_temp
$ sudo systemctl status get_temp
\end{verbatim}

Beim Systemstart mit starten:
\begin{verbatim}
$ sudo systemctl enable get_temp.service
\end{verbatim}

\end{frame}

\begin{frame}[fragile]
  \frametitle{Systemstart beschleunigen}

  Das folgende ist eine echte Analyse meines Arbeitsplatzrechners, kein Fake.
\begin{verbatim}
  $ systemd-analyze time
  Startup finished in 3.584s (kernel) + 20.245s (userspace)
  = 23.830s

  $ systemd-analyze blame
          9.533s NetworkManager-wait-online.service
          7.600s docker.service
          5.014s apt-daily.service
          3.804s apt-daily-upgrade.service
          1.546s exim4.service
          1.192s snapd.autoimport.service
           988ms ModemManager.service
           921ms dev-sda1.device
           843ms smbd.service
           ...
\end{verbatim}
\end{frame}

\begin{frame}[fragile]

  Startprozess als Grafik exportieren:
  \begin{verbatim}
  $ systemd-analyze plot > startup_plot.svg
  \end{verbatim}
  \includegraphics[height=0.6\linewidth]{Bilder/startup_plot.png}

\end{frame}

\begin{frame}[fragile]
Wie man sieht, wird sehr lange auf das Netwerk gewartet, was ich hier am Rechner aber nicht brauche:

\begin{verbatim}
$ sudo systemctl disable NetworkManager-wait-online.service
\end{verbatim}

Nach dem Neustart:
\begin{verbatim}
$ systemd-analyze time
Startup finished in 4.147s (kernel) + 8.690s (user) = 12.837s

$ systemd-analyze blame
        6.785s docker.service
        5.505s nmbd.service
        ...
\end{verbatim}
\end{frame}

\begin{frame}[fragile]
Was ist denn mit docker los?

\begin{verbatim}
$ sudo docker info
  Containers: 120
  ...
  Images: 488
\end{verbatim}

Sollte man 'mal aufräumen. Und samba benötige ich auch nicht: weg damit. Hinterher:

\begin{verbatim}
$ systemd-analyze blame
    2.090s docker.service
    2.022s snapd.autoimport.service
    1.075s exim4.service

$ systemd-analyze time
Startup finished in 4.183s (kernel) + 3.920s (user) = 8.103s
\end{verbatim}

Fazit: Startup von 23.8s auf 8.1s minimiert

\end{frame}

\begin{frame}
\frametitle{Fragen?}
  \begin{center}
  \Large{
    section77@josoansi.de \\~\\

    oder kommen Sie doch einfach zu unserem regelmäßigen Treffen, \\
    jeden 1. Mittwoch im Monat ab 20:00 Uhr. \\
    (Treffpunkt und Thema laut Webseite)
    }
  \end{center}
  \begin{figure}[ht]
    \centering
    \includegraphics[width=0.2\textwidth]{../gemeinsam/CC-BY-SA.png}
  \end{figure}
\end{frame}
\end{document}
