%%%%%%%%%%%%%%%%%%%%%%%%%%%%%%%%%%%%%%%%%
% Beamer Präsentation
% LaTeX Vorlage
% Version 1.0 (10/11/12)
%
% Diese Vorlage wurde heruntergeladen von:
% http://www.LaTeXTemplates.com
%
% Übersetzt und an eigene Bedürfnisse angepasst von Edgar 'Fast Edi' Hoffmann (08.03.2013)
% Lizenz:
% CC BY-NC-SA 3.0 (http://creativecommons.org/licenses/by-nc-sa/3.0/)
%
%%%%%%%%%%%%%%%%%%%%%%%%%%%%%%%%%%%%%%%%%

%----------------------------------------------------------------------------------------
%	PAKETE UND THEMEN
%----------------------------------------------------------------------------------------

\documentclass{beamer}

\mode<presentation> {

% Die Beamer-Klasse wird mit einer Reihe von Standard Folienthemen,
% mit unterschiedlichen Farben und Folienlayouts. Nachfolgend eine Liste aller verfügbaren Themen.
% Hier kann das jeweils gewünschte Thema unkommentiert werden.

%\usetheme{default}
% \usetheme{AnnArbor}
% \usetheme{Antibes}
% \usetheme{Bergen}
% \usetheme{Berkeley}
% \usetheme{Berlin}
% \usetheme{Boadilla}
% \usetheme{CambridgeUS}
% \usetheme{Copenhagen}
% \usetheme{Darmstadt}
% \usetheme{Dresden}
% \usetheme{Frankfurt}
% \usetheme{Goettingen}
% \usetheme{Hannover}
% \usetheme{Ilmenau}
% \usetheme{JuanLesPins}
% \usetheme{Luebeck}
 \usetheme{Madrid}
% \usetheme{Malmoe}
% \usetheme{Marburg}
% \usetheme{Montpellier}
% \usetheme{PaloAlto}
% \usetheme{Pittsburgh}
% \usetheme{Rochester}
% \usetheme{Singapore}
% \usetheme{Szeged}
% \usetheme{Warsaw}

% Neben verschiedenen Themen kommt die Beamer-Klasse auch mit diversen Farbvarianten
% für jedes Folienthema daher. Hier kann das jeweils gewünschte Farbthema unkommentiert werden.

% \usecolortheme{albatross}
% \usecolortheme{beaver}
% \usecolortheme{beetle}
% \usecolortheme{crane}
\usecolortheme{dolphin}
% \usecolortheme{dove}
% \usecolortheme{fly}
% \usecolortheme{lily}
% \usecolortheme{orchid}
% \usecolortheme{rose}
% \usecolortheme{seagull}
% \usecolortheme{seahorse}
% \usecolortheme{whale}
% \usecolortheme{wolverine}

%----------------------------------------------------------------------------------------
%	DEFINITION BENUTZERDEFINIERTER FARBEN (RGB)
%----------------------------------------------------------------------------------------
\definecolor{dunkelgrau}{rgb}{0.8,0.8,0.8}
\definecolor{hellgrau}{rgb}{0.95,0.95,0.95}
\definecolor{hellblau}{rgb}{0.8,0.85,1}
\definecolor{dunkelblau}{rgb}{0,0,0.9}

% \setbeamertemplate{footline}				% Zum Entfernen der Fusszeile auf allen Folien kann diese Zeile unkommentiert werden
\setbeamertemplate{footline}[frame number]		% Zum Ersetzen der Fußzeile mit einem einfachen Folienzähler kann diese Zeile auskommentiert werden

\setbeamertemplate{navigation symbols}{}		% Zum Anzeigen der Navigationssymbole am unteren Rand aller Folien kann diese Zeile auskommentiert werden
}

\setbeamertemplate{background canvas} [vertical shading][top=hellblau, bottom=white] 

% \usepackage{pgfpages}				% Zum Erstellen vernünftiger Ausdrucke/Handouts diese Zeile(n) unkommentieren
% \pgfpagesuselayout{resize to}[a4paper,border shrink=5mm,landscape]	% Einseitiger Ausdruck auf A4
% \pgfpagesuselayout{2 on 1}[a4paper,border shrink=5mm]			% Druckt 2 Folien auf 1 Seite A4
% \pgfpagesuselayout{4 on 1}[a4paper,border shrink=5mm,landscape]		% Druckt 4 Folien auf 1 Seite A4
% \pgfpagesuselayout{8 on 1}[a4paper,border shrink=5mm]			% Druckt 8 Folien auf 1 Seite A4
% \pgfpagesuselayout{16 on 1}[a4paper,border shrink=5mm,landscape]		% Druckt 16 Folien auf 1 Seite A4

% \setbeameroption{show notes on second screen=right}	% Zum Anzeigen von Notizen auf dem 2. Schirm.  Mit <location> = left, right, bottom oder top
									% Benötigt das Paket pgfpages!

%\beamerdefaultoverlayspecification{<+->}		% Aufzählungen immer schrittweise zeigen

% \setbeamercovered{transparent}			% Sorgt dafür, daß versteckte Textteile nicht unsichtbar, sondern mit wenig Kontast dargestellt werden

\setbeamerfont{frametitle}{family=\sffamily,series=\bfseries,size={\fontsize{18.00}{20.00}}}	% Setzen der Schriftart und -größe für die Überschriften
\setbeamerfont{footline}{family=\sffamily,series=\bfseries,size={\fontsize{6.00}{7.00}}}		% Setzen der Schriftart und -größe für die Fusszeile

\usepackage[ngerman]{babel}				% für Deutsch
%\usepackage{ngerman}
%\usepackage[english]{babel}				% für Englisch
%\usepackage[latin1]{inputenc}
\usepackage[utf8]{inputenc}
\usepackage{graphicx}					% Erlaubt das Einbinden von Grafiken/Bildern
\usepackage{eurosym}					% Erlaubt das Anzeigen des Eurosymbols
\usepackage{xcolor}					% Erlaubt den Einsatz benutzerdefinierter Farben
								% Weitere Einsatzmöglichkeiten:
								% \textcolor{Farbe}{Text} entspricht Befehlen, wie \textbf{}
								% \pagecolor{Name}, \pagecolor{model}{specification} setzt die Hintergrundfarbe der Seite
								% \colorbox{Name}{Text}, \colorbox{model}{specification}{Text} hinterlegt den Text mit der Farbe
								% \fcolorbox{Name}{Text}, \fcolorbox{model}{specification}{Text} umrandet den Text mit der Farbe
								% vordefinierte Farben: black, white, red, green, blue, cyan, magenta, yellow

\usepackage{colortbl}					% Erlaubt das Verwenden benutzerdefinierter Farben in Tabellen
								% \cellcolor{farbe}, \rowcolor{farbe} oder \columncolor{farbe}

\usepackage{url}						% Erlaubt das Verwenden von URLs ohne umständliche Maskierung
\urlstyle{tt}							% Durch Verwendung von \urlstyle{tt} kann zusätzlich die Schriftart auf Typewriter gestellt werden
\urldef{\fsog}{\url}{http://www.FreieSoftwareOG.org}	% Häufig vorkommende URLs können in einem Makro hinterlegt werden.

\usepackage{booktabs}					% Erlaubt den Einsatz von \toprule, \midrule und \bottomrule innerhalb von Tabellen

% \usepackage{fontspec}					% Erlaubt das Anpassen von Schriftarten. Z.B. auch um eigene Fonts zu verwenden. MUSS MIT XELATEX KOMPILIERT WERDEN!!
% \defaultfontfeatures{Mapping=tex-text}
% \setmainfont{Yanone Kaffeesatz}			% (Haupt-)Schriftart der Präsentation

% Will man ein globales Hintergrundbild verwenden, muss dies in der Preamble definiert werden
% Hier verwende ich das der FSOG
%\usebackgroundtemplate%
%{
%  \includegraphics[width=\paperwidth,height=\paperheight]{../gemeinsam/Praesentationshintergrund}%
%}

\usepackage{pgf}  
\logo{\pgfputat{\pgfxy(-1.2,-0.2)}{\pgfbox[center,base]{\includegraphics[width=2.6cm]{../gemeinsam/fsog.png}}}}  

%\logo{\includegraphics[width=3cm]{../gemeinsam/fsog.png}}




%----------------------------------------------------------------------------------------
% DEFINITION DER TITELFOLIE
%----------------------------------------------------------------------------------------

% Der Kurztitel (in eckiger Klammer) erscheint im unteren Bereich aller Folien, der Haupttitel (in geschweifter Klammer) erscheint nur auf der Titelfolie
\title[FreieSoftwareOG.org - Videobearbeitung - Ein Überblick]{Videobearbeitung - Ein Überblick} 
\author{Edgar 'Fast Edi' Hoffmann}    % Name des Vortragenden
\institute[FSOG]          % Die Institution. Erscheint am unteren Rand aller Folien, zum Platz sparen möglichst eine Abkürzung
{
  Community FreieSoftwareOG \\      % Die Institution. Erscheint auf der Titelfolie
  \medskip
  \textit{kontakt@freiesoftwareog.org}    % Die Mailadresse. Erscheint auf der Titelfolie
}
\date{\today}         % Das Datum des Vortrags (hier: aktuelles Datum)

%----------------------------------------------------------------------------------------
% ANFANG DES DOKUMENTES / DER PRÄSENTATION
%----------------------------------------------------------------------------------------

\begin{document}

%\begin{frame}[plain]      % Der Zusatz [plain] sorgt dafür, daß keine Seitenzahl/Fusszeile verwendet wird
\begin{frame}      % Der Zusatz [plain] sorgt dafür, daß keine Seitenzahl/Fusszeile verwendet wird
  \titlepage        % Ausgabe der Titelfolie als erste Folie
\end{frame}

%------------------------------------------------------------------------------------------
% AB HIER BEGINNT DIE EIGENTLICHE PRÄSENTATION
%------------------------------------------------------------------------------------------

\begin{frame}
\frametitle{Videobearbeitung - Begriffserklärung}

\pause
Mit der Videobearbeitung wird das Ziel verfolgt, bewegte Bilder zu erstellen, die sich von dem Ausgangsmaterial unterscheiden.

\end{frame}

\begin{frame}

\frametitle{Wozu Videos bearbeiten?}

\pause
  \begin{itemize}
    \item Entfernen von Werbeblocks
    \item Verbessern des Ausgangsmaterials
    \item Zusammenschneiden einzelner Sequenzen
    \item Übergänge, Audio- oder Textkommentare hinzufügen
  \end{itemize}
\end{frame}

\begin{frame}
  \frametitle{Videoformate}
Videoformat bezeichnet in der Videotechnik die Zusammenfassung audiovisuellen Spezifikationen eines Videos:\pause
  \begin{itemize}
    \item Bildauflösung und das abgeleitete Seitenverhältnis
    \item Farbtiefe
    \item Bildwiederholungsrate
    \item Tonspur
  \end{itemize}
\end{frame}

\begin{frame}
\frametitle{Videoschnitt-Tools}
Ausser direkten Videoschnitt-Programmen kann man auch andere Tools als Hilfsprogramme einsetzen:\pause
  \begin{itemize}
    \item Blender (Linux, Windows)
  \end{itemize}
\end{frame}

\begin{frame}
  \frametitle{Videoschnitt-Software:\\Offline}
  \pause
  \begin{itemize}
    \item Avidemux (Linux, Windows, Mac)
    \item VirtualDub (Windows)
    \item Cinefx (Linux, Windows, Mac)
    \item Cinelerra (Linux)
    \item Lightworks (Linux, Windows)
    \item Kdenlive (Linux, Mac)
    \item Kino (Linux)
    \item LiVES (Linux)
    \item Openshot (Linux)
    \item VideoLAN VLMC (Linux, Windows, Mac)\\...bisher nur angekündigt...
  \end{itemize}
\end{frame}

\begin{frame}
  \frametitle{Videoschnitt-Software:\\Online}
  \pause
  \begin{itemize}
    \item Youtube Editor
    \item Videnio - Online-Videobearbeitungsportal (deutsch)
  \end{itemize}
\end{frame}

\begin{frame}[fragile]   % Die "fragile"-Option muss verwendet werden, wenn auf der Folie verbatim verwendet wird
\frametitle{Links zur Präsentation}
  \begin{verbatim}
    http://www.blender.org
    http://www.avidemux.org
    http://www.virtualdub.org
    http://sourceforge.net/projects/cinefx/
    http://www.cinelerra.org
    http://www.kdenlive.org
    http://www.kinodv.org
    http://lives.sourceforge.net/
    http://www.openshot.org
    http://www.videolan.org
  \end{verbatim}
\end{frame}

\begin{frame}
\frametitle{Weitere Informationen bekommen Sie hier:}
  \begin{center}
  \Large{
    \fsog \\      % Makro für das Einfügen der URL (oben definiert)
    und \\
    Kontakt@FreieSoftwareOG.org \\~\\

    oder kommen Sie doch einfach zu unserem regelmäßigen Treffen, \\
    jeden 1. Mittwoch im Monat ab 20:00 Uhr. \\
    (Treffpunkt und Thema laut Webseite)
    }
  \end{center}
  \begin{figure}[ht]
    \centering
    \includegraphics[width=0.2\textwidth]{../gemeinsam/CC-BY-SA.png}
  \end{figure}
\end{frame}

\end{document}
